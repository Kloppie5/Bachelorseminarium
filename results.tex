\section{Literature Review}
In this section we'll answer all (sub-)research questions to get an overview of the current state-of-the-art of Distributed Machine Learning:
\begin{itemize}
	\item Why do we want or need Distributed Machine Learning opposed to (centralized) Machine Learning?
	\item What technology is used for Distributed Machine Learning?
	\item What are the currently used implementations?
\end{itemize}

\subsection{The need for Distributed Machine Learning}
.
\section{Alternatives to Distributed machine learning}
Even though the most popular solution to increasing computation power for Machine
learning is currently distributing the workload over a large amount of machines
there are other more traditional ways to increase computation power. These ways
being the use of Application Specific Integrated Circuits, and special multi core
computer architectures.

\subsection{Asics}
The idea of creating ASICs for use in highly specialized tasks is not a new idea
to machine learning, in recent times the demand for such chips has risen masively[5]
looking at applications such as bitcoin mining asics have had a massive competitive
advantage over GPUS and CPUS rendering them essentially useless. The main reason
that ASICs have such a good outlook when looking at machine
learning is the fact that most of the work in most areas are matrix multiplications.
This means that it is viable to have a chip that can only multiply matrixes and nothing
else.

Google took this idea of ASICs and have created their Tensor Processing Unit (TPU)[1]
which as the name hints is an ASIC computer that specializes in calculations that
use tensors, this works well together with their Tensor flow library which is one
of the most popular Neural Network Libraries around at the moment. Architecture of
the chip can be seen in figure \ref{TPU_Architecture}.

\begin{figure}
  \includegraphics[scale=0.5]{TPU_Architecture.png}
  \caption{Google TPU Architecture[4]}
  \label{TPU_Architecture}
\end{figure}

The most important aspect of the TPU is the Matrix Multiply unit that is positioned
on the right of the diagram. As can be seen the TPU uses a PCIe bus to communicate
with a server, so it is not a stand alone unit. Which is usefull as it can then also
be used in a distributed setting.

The performance improvement of the TPU vs a regular CPU/GPU combinations is not only
in the increased processing power, but also in the efficiency, which is important
for big companies that want to run many TPUs in the same building. In figure \ref{TPU_Relative_Performance}
it can be seen that the performance/watt relative to GPUs and CPUs of a TPU is many times higher,
up to nearly 200x the performance/watt.

\begin{figure}
  \includegraphics[scale=0.5]{TPU_Relative_Performance.png}
  \caption{TPU performance, relative to CPU and GPU, the total performance includes server power costs. GM and WM are geometric and weighted means respectively [4]}
  \label{TPU_Relative_Performance}
\end{figure}

Other than the power efficiency the total processing power of a TPU is also far higher
than a GPU or CPU in figure \ref{TPU_Performance} where it can be seen that the pure
power of a TPU far outscales a CPU/GPU in most cases.

\begin{figure}
  \includegraphics[scale=0.5]{TPU_Performance.png}
  \caption{TPU performance on 6 common machine learning tasks[1]}
  \label{TPU_Performance}
\end{figure}


\subsection{Computer architectures}
Other than using ASICs in order to increase the amount of work a computer can do
a different architecture can also be used that is deisgned to increase the amount
of useable cores on a computer massively. Such an architecture exists in the form
of the Epiphany architecture. The Epiphany architecture is a Multiple Instruction,
Multiple Data (MIMD) architecture that uses an array of processors, that all use the
same memory to speed up execution of floating point operations[2].

The newest chip that Adapteva have managed to produce is the Epihpany V, which uses
1024 cores on a single chip[2]. Although the power consumption of their newest chip
is not yet public information, they have released numbers boasting a power usage of
only 2 watt on their Epiphany IV chip[6].

Using multiple processors to process one set of data is not a new idea in computer
Science, for this purpose MPI was developed to simplify the distribution of work,
usually over a cluster of workers. The same framework can also be adapted to the
Epiphany chips as was demonstrated by Richie et al. in their paper. They found that
the RISC chips were able to perform extremely fast matrix operations until their on board
memory was filled[3]. in figure \ref{Epiphany_MPI} it can be seen that performance increases
until the matrix no longer fits in memory.

\begin{figure}
  \includegraphics[scale=0.5]{Epiphany_MPI.png}
  \caption{Epiphany chip performance with different matrix sizes[3].}
  \label{Epiphany_MPI}
\end{figure}

\subsection{Conclusion}
As can be seen with the examples above there are many differnt ways in which the
processing power needed for machine learning can be create. With both focussing
on power efficiency as well as computational power. However it must be noted,
that due to the benefits of space and power supply offered by a distributed system
all alternative solutions, work alot better than a normal computer when not distributed
yet they all have the ability to be used together in a distribued setting without
loosing much if any in the form of performance.


[1] https://cloud.google.com/blog/big-data/2017/05/an-in-depth-look-at-googles-first-tensor-processing-unit-tpu %google tpu bibliography
[2] https://arxiv.org/pdf/1610.01832.pdf %epiphany V chip
[3] https://www-sciencedirect-com.tudelft.idm.oclc.org/science/article/pii/S1877750315000617?_rdoc=1&_fmt=high&_origin=gateway&_docanchor=&md5=b8429449ccfc9c30159a5f9aeaa92ffb&ccp=y %mpi on epiphany architecture
[4] http://delivery.acm.org/10.1145/3090000/3080246/p1-Jouppi.pdf?ip=145.94.160.220&id=3080246&acc=OA&key=0C390721DC3021FF%2E512956D6C5F075DE%2E4D4702B0C3E38B35%2E5945DC2EABF3343C&__acm__=1521207947_dd71d367ebbdf28d05c9b10548e1c130 %tpu vs cpu and gpu
[5] https://www.nytimes.com/2018/01/14/technology/artificial-intelligence-chip-start-ups.html %rise in demand for chips
[6] http://www.adapteva.com/docs/e64g401_datasheet.pdf










\subsection{Underlying technology}
To give you an overview of how Distributed Machine Learning works, we'll give you an abstract framework that includes everything a real implementation should include. We do that by exploiting the three unique properties of Distributed Machine Learning, namely error tolerance, dynamic structural dependency and non-uniform convergence.\cite{Xing16}\\
Our goals include (1) to list regular Machine Learning algorithms that are commonly used in a distributed setting; (2) to find algorithms to determine the best parameters for the former algorithms; (3) to trade off computation time with communication and accuracy; and (4) to minimize the amount of bits sent over the network so that the system is no bottlenecked by scarce network bandwidth.\\
Designing a general system in such a way that the regular Machine Learning algorithms can be distributed efficiently is challenging because every algorithm has its own communication patterns \cite{Jia14}\cite{Newman09}\cite{Rich13}\cite{Smola10}\cite{Takac13}\cite{Tsi12}.

\subsubsection{Machine Learning algorithms}
Machine learning algorithms learn to make predictions or decisions based on data. Every machine learning algorithm has its own pros and cons. Machine learning algorithms can be divided into categories based on some of their characteristics:
\begin{itemize}
	\item \textbf{Feedback}, the type of feedback given to the algorithm during its learning
	\item \textbf{Goal}, the type of output
	\item \textbf{Type}, the way they create their output
	\item \textbf{Method}, the way they change when given feedback
\end{itemize}

\paragraph{Feedback}
\begin{itemize}
	\item \textbf{Supervised}
		learning uses training data which consists of input objects, usually vectors, and output values. Supervised learning algorithms typically try to find a function that maps input data to some output and use that function to find outputs for unknown input data.
		% bias vs variance
		% complexity vs amount of data
		% dimensionality
		% noise
	\item \textbf{Unsupervised}
		learning uses training data without labels, which means that there is no way to determine the accuracy of the output. It is commonly used to find clusters of data points or patterns.
	\item \textbf{Semi-supervised}
		learning makes use of usually a small amount of labeled data and a large amount of unlabeled data. Clustering can be used to extend the labels known to other data points. This is done under the assumption that similar data points share a label.
	\item \textbf{Reinforcement learning}
		is relies on letting the model generate an output and letting a different system give feedback on how well the output matches the desired output. This is done by either minimizing the risk function or by maximizing the reward function. This means that reinforcement learning can be done without having a big dataset of correct and incorrect input-output pairs. Because of the black box behavior of the model, it is not possible to explicitly correct actions that are sub-optimal and the model might get stuck in a local minimum.
		% Q-learning
\end{itemize}

\paragraph{Goals}
\begin{itemize}
	\item \textbf{Anomaly detection}
		can be divided into 3 categories. \textbf{Supervised anomaly detection} requires data that has a different label for abnormal data and trains with this data a normal classifier. \textbf{Unsupervised anomaly detection} assumes that "normal" and "abnormal" data points will be separate groups and clusters these. \textbf{Semi-supervised anomaly detection} builds a model of normal data and tests whether new data points are normal or abnormal. Based on the dataset, different methods perform better than others
	\item \textbf{Classification}
		is the problem of putting new data points in categories based on the categories of the training data. This is an inherently supervised process. The unsupervised version of this is clustering.
	\item \textbf{Clustering}
		is the problem of grouping together data points that are similar according to some criteria. This can be done with a supervised implementation, but is usually done with significantly large datasets for which labeling might be too expensive.
	\item \textbf{Dimensional reduction}
		is the problem of reducing the amount of variables used. This can be done by either selecting only the relevant variables (which is called \textbf{Feature selection}), or by creating new variables that represent multiple other variables (which is called \textbf{Feature extraction}).
	\item \textbf{Feature learning / Representation learning}
		is the set of techniques designed for finding good representations of the data for things like feature detection, classification, clustering, encoding and even matrix factorization.
		% manifold learning
		% statistical inference/density estimation
		% sparse coding
	\item \textbf{Regression}
		is the problem of estimating how a dependent variable changes with changes to the independent variables. Parametric regression tries to find the parameters of a function, which speeds up the process if for example a linear correlation is expected. Nonparametric regression constructs the predictor according to information derived from the data without relying on a predetermined form. This requires a larger sample size and significantly more time.
\end{itemize}

\paragraph{Types}
\begin{itemize}
	\item \textbf{Evolutionary algorithms (EAs)},
		specifically \textbf{Genetic algorithms (GAs)} are algorithms that learn iteratively based on evolution. The algorithm that actually solves the problem is represented by a set of data that represents its properties, called its \textbf{genotype}. The performance of the algorithm is measured using a score, calculated using a \textbf{fitness function}. After calculating the fitness score of all generated algorithms, the next iteration creates new genotypes based on mutation and crossover of algorithms that are produce more accurate estimates. Generic algorithms can be used to create other algorithms like neural networks, belief networks, decision trees and rule sets.
	% Graphical models
	\item \textbf{Perceptron-based}
		algorithms are based on perceptrons, binary classifiers that map an input vector to being 'active' or 'inactive' by assigning a weight to all inputs, summing over the products of these weights and their inputs, and comparing these to a threshold which is called the \textbf{bias}. Perceptron-based algorithms commonly use the entire batch of training data to try to find an instance that will work for the entire set. Perceptron-based algorithms are binary and are therefore mainly used for binary or multi-label classification.
	\item \textbf{Rule-based machine learning (RBML)}
		algorithms use a set of rules that represent a small part of the problem each. These rules usually express a condition and a result for when that condition is met. Because of this if-then relation, rules lend themselves to be interpreted more easily than relatively abstract types of ML algorithms such as neural networks.
	\item \textbf{Topic Models (TM)}
		are statistical models for finding and mapping semantic structures in text.
\end{itemize}

\paragraph{Methods}
\begin{itemize}
	\item \textbf{Association rule learning}
		is a \textit{rule-based machine learning} method that focuses on finding relations between different variables in datasets. This is done by looking at some measure of interest. Examples of this are \textbf{Support} (how often variables appear together), \textbf{Confidence} (how often a causal rule is true) and \textbf{Collective Strength} (a comparison of the amount of instances which contains some of the variables in the relation and the expected amount of these instances if the variables where not related).
	\item \textbf{Artificial neural networks (ANNs)}
		are perceptron-based systems using multiple layers. These layers are usually divided into an input layer, an output layer, and one or more hidden layers. Each layer consists of nodes connected to the previous and next layer by edges with associated weights, usually called synapses. Unlike "normal" perceptrons, these nodes usually use an activation function instead of just a bias.\\
		The workings of the algorithm is dependent on the entire network, the algorithm can be changed by changing (1) the weights of the synapses, (2) the layout of the network or (3) the activation function of nodes.\\
		Because neural networks require a big amount of nodes, the understandability of what a neural network actually does is a major drawback when comparing it to for example decision trees.\\
		Neural networks are extensively studied for their ability to analyze enormous sets of data. They can be divided into subgroups based on the layout of the network;
		\begin{itemize}
			\item \textbf{Recurrent neural networks (RNNs)}
				have synapses going back to previous layers, which means that the previous state of the network influences its current decisions. Neural networks that are not recurrent are called \textbf{feed-forward networks}. Recurrent synapses give the network a sort of memory that can help with discovering patterns in data that arrives sequentially. Special blocks of nodes in a network can work as a memory cell that can hold some information for an arbitrarily long timespan. These blocks are called \textbf{Long Short-Term Memory (LSTM)} units.
				% finite impulse
				% 	DAG which can become a feedforward network 
				% vs infinite impulse
				% 	DAG which can not be unrolled
			\item \textbf{Hopfield networks}
				are a type of non-reflexive, symmetric recurrent neural network that have some 'energy' related to every state of the network as a whole which will reach a local minimum after continuously updating the network.
			\item \textbf{Deep neural networks (DNNs)},
				are the opposite of \textbf{shallow neural networks} and have many hidden layers which may cause the network to work as a sort of a black-box.
			\item \textbf{Convolutional neural networkss (CNNs / ConvNets)}
				are deep, feed-forward neural networks that use convolution layers with nodes connected to only a few nodes in the previous layer. These values are then pooled using pooling layers that can be seen as a way of recognizing abstract features in the data. The convolution makes the network look at local data, making the entire algorithm spatially invariant, which is why they are sometimes called Space Invariant Artificial Neural Networks (SIANN). Chaining multiple of these convolution and pooling layers together can make the network very good at recognizing complicated constructs in big datasets, like recognizing cars in images or the contextual meaning of a sentence in a paragraph.
			\item \textbf{Generative adversarial networks (GANs)}
				consist of two separate networks, one trying to recognize objects from a dataset and one trying to create new data in an attempt to 'fool' the other network into thinking the data is legit.\cite{Li:2013:CAL:2463372.2465801}
			\item \textbf{Radial Basis Function (RBF) networks}
				\cite{rbf}
			\item \textbf{Self-organizing maps (SOMs) / self-organizing feature maps (SOFMs)}
				are neural networks that learn by unsupervised \textbf{competitive learning} in which nodes compete for access to specific inputs, causing the nodes to become highly specialized, which reduces redundancy. The iterations effectively move the map closer to the training data, which is the reason for its name. Some subtypes include the Time Adaptive Self-Organizing Map (TASOM), Binary Tree TASOM (BTASOM) and Growing Self-Organizing map (GSOM)
			\item \textbf{Stochastic neural networks}
				make use of stochastic transfer functions or weights which makes it possible to escape a local minimum. 
			% 		boltzmann machine
		\end{itemize}
		Neural networks are trained in many different ways, like by using genetic algorithms. The most common approach, especially when talking about Deep neural networks, is using a process called \textbf{backpropagation};
		\begin{itemize}
			\item Present a training sample
			\item Calculate the error in each output neuron, how much lower or higher the output must be adjusted to match the desired output
			\item Calculate the gradient
			\item For every layer, adjust the weights and biases based on the gradient
			\item Repeat
		\end{itemize}
	\item \textbf{Auto-encoders}
		are a type of neural network that trained specifically to encode and decode data. Because auto-encoders are trained to perform decoding separately from encoding, the encoded version of the data can be seen as a dimensional reduction of the data
	\item \textbf{Bayesian networks}, 
		sometimes called \textbf{belief networks}, are probabilistic directed acyclic graphical models\cite{Wain08}\cite{Kol09}\cite{Xin16} that are used to represent conditional relationships between variables. They are directed to overcome the problems of \textbf{Markov random
		fields (MRFs)}, also called \textbf{Markov networks}, which use undirected connections. These undirected connections make it impossible to represent dependencies that are non-transitive or otherwise induced. Bayesian networks are commonly used to represent \textbf{Markov processes} (not related to Markov networks), in particular for probabilistic inference or parameter estimations.
		\textbf{Nonparametric Bayesian models}\cite{Grif05}\cite{Teh06} don't fix the parameters in place, allowing the model to grow with the data size.
	%		Regularized Bayesian models \cite{Zhu09}\cite{Zhu09-2}\cite{Zhu14}
	\item \textbf{Decision trees},
		sometimes called CART trees, after Classification And Regression Trees (RAS syndrome), use rule-based machine learning to create a set of rules to create decision branches. Traversing the tree means applying the rules at each step until you arrive at a leaf of the tree, which represents the decision or classification for that input.
	\item \textbf{Latent Dirichlet Allocation}\cite{Blei03}
		makes a mapping between documents and probabilistic set of topics, using the assumption that documents have few different topics and that those topics use few different words.
	\item \textbf{Latent semantic analysis (LSA) / latent semantic indexing (LSI)}
		creates a big matrix of documents and topics in an attempt to classify documents or to find relations between topics. LSA/LSI assumes a Gaussian distribution for topics and documents. LSA/LSI doesn't have a way of dealing with words that have multiple meanings.
	\item \textbf{Naive Bayes classifiers}
		are relatively simple probabilistic classifiers that assume different features to be independent. They can be trained quickly using supervised learning, but are less accurate than more complicated approaches.
	\item \textbf{Probabilistic latent semantic analysis (PLSA) / probabilistic latent semantic indexing (PLSI)}
		is the same as LSA/LSI except that PLSA/PLSI assumes a Poisson distribution for topics and documents instead of the Gaussian distribution that is assumed by LSA/LSI. The reason is that a Poisson distribution appears to model the real world better. Some subtypes include Multinomial ASymmetric Hierarchical Analysis (MASHA), Hierarchical Probabilistic Latent Semantic Analysis (HPLSA) and Latent Dirichlet Allocation (LDA).
	% Inductive logic programming (ILP)
	\item \textbf{Support vector machines (SVMs)}
		map data points to high dimensional vectors for classification and clustering purposes. For data points in a p-dimensional space, a (p-1)-dimensional hyperplane can be used as a classifier. A reasonable choice would be the hyperplane that properly separates the data points in two groups based on their labels and also has the biggest possible margin. Sometimes special transformation equations called \textbf{kernels}, are used to change all data points to a different representation in which it is easier to find such a hyperplane.\\
		Having a lot of dimensions decreases the accuracy of classifying new data points, so increasingly large datasets are needed to make the algorithms perform well.
\end{itemize}

Usually a single algorithm isn't accurate enough to solve a problem and therefore multiple algorithms are combined in so-called \textbf{Ensemble Learning}. There are many different ways to do this, for example:

\begin{itemize}
	\item \textbf{Bagging}
		is the process of building multiple classifiers and combining them into one.
	\item \textbf{Boosting}
		is the process of training new models with the data that is misclassified by the previous models.
	\item \textbf{Bucketing}
		is the process of training many different models and eventually selecting the one that has the best performance.
	\item \textbf{Random Forests}\cite{Breiman2001}
		use multiple decision trees and averaging the prediction made by the individual trees as to increase the overall accuracy. Different trees are given the same 'voting power'.
	\item \textbf{Stacking}
		is when multiple classifiers are trained on the dataset and one new classifier uses the output of the other classifiers as input in an attempt to reduce the variance.
	\item \textbf{Learning Classifier Systems (LCSs)}
		(not to be confused with the more common usage of the LCS initialism, Longest Common Subsequence) is a kind of a modular system of learning approaches. An LCS iterates over data points from the dataset and completes the entire learning process in each iteration. The main idea is that an LCS has a limited number of rules that follow a Genetic Algorithm that forces rules that are suboptimal out of the ruleset. There are many different attributes that can drastically change the performance of the LCS based on the dataset including:
		\begin{itemize}
			\item Michigan-style architecture vs Pittsburgh-style architecture
			\item supervised vs reinforcement
			\item incremental vs batch
			\item online vs offline
			\item strength-based vs accuracy-based
			\item complete mapping vs best mapping
		\end{itemize}
		Because of the increasing complexity of the training of LCS with respect to the size of the data, LCS is the most commonly used implementation.
\end{itemize}


%Deep learning
%
%Multi-task learning (MTL) 
%
%Matrix factorization
%Singular value decomposition (SVD) and principal component analysis (PCA)


\paragraph{Algorithms to find best parameters}
To find the parameters for these algorithms we can use several Machine Learning workhorse implementations that can be re-used across different Machine Learning algorithm-families. These include:
\begin{itemize}
	\item First-order techniques
	\begin{itemize}
		\item Stochastic gradient descent
% 			Backpropagation
		\item Stochastic dual coordinate ascent\cite{Shal13}
		\item L-BFGS
		\item Conjugate gradient
	\end{itemize}
	\item Second-order techniques
	\begin{itemize}
		\item Newton descent
		\item Quasi-Newton descent
	\end{itemize}
	\item Coordinate descent
	\item Markov-Chain Monte-Carlo
	\item Variational inference
\end{itemize}


% --------


\subsubsection{Partitioning and distribution algorithms}
Now we'll look the abstraction of the general design of a Distributed Machine Learning operating system that executes the Machine Learning workhorses across a wide variety of hardware.


\paragraph{Computation time vs communication vs accuracy}


\paragraph{Scheduling and balancing the workloads}
There are 3 things to take into account when partitioning an ML program in order to parallelize it:\cite{Xing16}\\
\begin{itemize}
	\item Deciding which tasks go or don't go together in parallel
	\item Deciding the order in which tasks will be executed
	\item Ensuring an even load distribution across the machines
\end{itemize}


\paragraph{Bridging computation and communication}
To efficiently communicate between the nodes there are several techniques that take the interleaving of parallel program computations and inter-worker communication into account. These techniques trade off fast / correct model convergence (at the top of the list) with faster / fresher updates (at the bottom of the list).
\begin{itemize}
	\item \textbf{Bulk Synchronous Parallel (BSP):} the most simple model; programs will alternate between a computation and a communication phase to ensure consistency\cite{Xing16}. An example of program following the BSP bridging model is MapReduce.\\
	An advantage is that serializable BSP ML programs are guaranteed to output a correct solution. A disadvantage is that finished workers must wait at every synchronization barrier till the other works are finished, which results in overhead\cite{Chilimbi14}. Another disadvantage is that the synchronization barrier may take a significant amount of time, because of slow communication between the workers.
	\item \textbf{Stale Synchronous Parallel (SSP):} allows the fastest worker to be ahead of the slowest worker up to a bounded number of iterations before the workers are stopped. That way, the fast workers may work for a while on state data, but SSP still limits the maximum staleness between any pair of workers. An advantage is that it still enjoys strong model convergence guarantees. A disadvantage is that, when machines temporarily slow down due to other tasks or users and cause large staleness, the convergence rates are poor.
	\item \textbf{Approximate Synchronous Parallel (ASP):} limits how inaccurate a parameter can be, in contrast to SSP that limits how stale a parameter can be. An advantage is that, when an aggregated update is insignificant, then the server can delay synchronization indefinitely. A disadvantage is that it can be hard to choose the parameter that defines which update are significant and which are not. \cite{Hsieh17}
	\item \textbf{Barrierless Asynchronous Parallel\cite{Han15} / Total Asynchronous Parallel\cite{Hsieh17} (BAP / TAP):} worker machines communicate in parallel without waiting for each other. The advantage is that it usually obtains the highest speedup possible. A disadvantage is that the model converges slowly or even incorrectly because the error grows with the delay, unlike BSP and SSP. \cite{Han15}
\end{itemize}
At the moment, ASP is the state-of-the-art method.


\paragraph{Communication strategies}
To distribute Machine Learning algorithms we can choose to partition either the data or the model across the machines - referred to respectively as data parallelism and model parallelism \cite{Die12}. These two types of parallelism can be applied at the same time \cite{Xing16}. Data parallelism can be applied to every ML algorithm with an i.i.d. assumption over the data samples, which are most ML algorithms \cite{Xing16}. It partitions the data and assigns it to parallel machines. Model parallelism cannot be applied to most ML algorithms, because the model parameters generally don't have this i.i.d. assumption. It partitions the model parameters and assigns that to parallel machines.\\
There are several communication management strategies\cite{Xing16} used to spread and reduce the amount of data communicated between machines:
\begin{itemize}
	\item To prevent bursts of communication over the network, for example after a mapper is finished, continuous communication is used, for example in the state-of-the-art implementation Bösen\cite{Wei15}.
	
	\item Neural networks are composed out of layers, of which the training by using the back-propagation gradient descent algorithm is highly sequential. Because the top layers of neural networks contain a lot more parameters while it accounts only for a small part of the total computation \cite{Xing16}, WFBP \cite{Zhang17} was proposed to spread the computation and communication out in an optimal fashion.
	
	\item Hybrid communication (HybComm) Because WBFP does not reduce the communication overhead, HybComm \cite{Zhang17} was proposed. Effectively it combines Parameter Servers (PS)\cite{Wei15} with Sufficient Factor Broadcasting (SFB)\cite{Xie15} by being aware of both the mathematical property of neural networks and the structure of computing clusters. See below for more information about PS under Centralized Storage and SFB under Decentralized Storage.
	
\end{itemize}


\paragraph{Network topologies}
There are several network topologies used for Distributed Machine Learning clusters:
\begin{itemize}
	\item \textbf{Trees} AllReduce\cite{Agar14} is an example of a tree-like network topology that provides straightforward parallelization of gradient-based optimization algorithms by accumulating local gradients to obtain a global gradient.\\
	An advantage is that AllReduce is very fast and highly scalable. A disadvantage is that it only works for reduce operations like in MapReduce.
	\begin{minipage}{\linewidth}
		\centering
		\includegraphics[scale=0.5]{AllReduce.png}
		\captionof{figure}{AllReduce operation. Initially, each node holds its own value. Values are passed up the tree and summed, until the global sum is obtained in the root node (reduce phase). The global sum is then passed back down to all other nodes (broadcast	phase). At the end, each node contains the global sum.}
	\end{minipage}
	\item \textbf{Centralized storage} Centralized storage is used for topologies where a single "master" or "server" node is in the center and many "slave" or "client" nodes communicate only with this master node. A practical implementation of this is the Parameter Server paradigm (PS). Each Parameter Server keeps a shard of the global model parameters as a key-value store. Each client communicates with the Parameter Server to read / update the parameters. \\
	An advantage is that all model parameters are in a global shared memory which makes it easy to inspect the model. A disadvantage is that the Parameter Servers are a bottleneck because they're handling all communication. To partly accommodate this issue, the techniques mentioned under "Bridging computation and communication" are used.
	\item \textbf{Decentralized storage} In contrast to centralized storage, in decentralized storage every worker maintains its own local view of the parameters and the workers communicate directly to each other. An example implementation is a peer-to-peer network where every machine broadcasts everything to all other machines. To reduce the amount of communication between all workers, Sufficient Factor Broadcasting (SFB)\cite{Li13}) was proposed. It decomposes the parameter matrix into so-called sufficient factors, i.e. 2 vectors that are sufficient to reconstruct the update matrix. SFB only broadcasts the sufficient factors and lets the workers reconstruct the updates.\\
	An advantage is that, with SFB, decentralized storage is relatively communication efficient and that there is no centralized bottleneck, making it more scalable.
\end{itemize}










\subsection{Currently used implementations}
In this section we'll first look at generic implementations for distributed systems, because these are used as the foundation for distributed Machine Learning systems. For the latter one, we'll look at the most popular implementations, including Distributed Ensemble Learning, Parallel Synchronous SGD and Parameter Servers.

\subsubsection{Generic distributed system frameworks}
Distributed system have already been implemented in the industry to solve the problem of companies possessing massive amounts of data that they want to be able to query and analyze. These frameworks largely rely on the fact that it is cheaper
to have multiple servers, each of them with a relatively small storage capacity and computing power, rather than having one expensive large server.

\paragraph{The frameworks}
The basis of existing frameworks is based on the Google File System\cite{Ghem03}. The \textit{Google File System} or \textit{GFS} is the system that is used within Google to handle all the Big data needs within the company. It splits all the data that is uploaded to the cluster up into chunks, which are then split over the "chunk servers" in the node, and replicated a predefined number of times (usually 3) in order to guarantee that the data is not lost when a server fails. The
data on the chunk servers can then be accessed by a user by contacting the master, which knows exactly on which servers each chunk is saved.\cite{Ghem03} The data and work-flow is described in figure \ref{GFS_Architecture} below.

\begin{figure}
  \includegraphics[width=\textwidth]{GFS_Architecture.png}
  \caption{Google File System architecture\cite{Ghem03}}
  \label{GFS_Architecture}
\end{figure}

The GFS architecture that was described in its paper by Google was adapted into an open-source solution called Hadoop \cite{Shv10}. Hadoop was developed mainly by Yahoo!, distributed as an Apache project and functions essentially the same as GFS. There are only minor differences with GFS such as the naming of certain entities and the chunk size. A typical use case of adding a file to the Hadoop File System (HDFS) is show in figure \ref{Hadoop_usecase} below.

\begin{figure}
  \includegraphics[width=\textwidth]{Hadoop_use_case.png}
  \caption{The data flow when adding a file to the HDFS\cite{Shv10}}
  \label{Hadoop_usecase}
\end{figure}

\paragraph{Uses of the frameworks}
The storage framework highlighted above, is the most-common file-system that empower Big Data implementations today. These implementations are frameworks such as MapReduce and the Apache Spark engine.

\paragraph{MapReduce}
MapReduce is a new framework for processing data and was developed by Google\cite{Dean04} in order to process data in a distributed setting. Firstly, in the \textit{map phase} all data is split into tuples (called key-value pairs). Then, during the shuffle phase, these key-value pairs are shuffled and passed to the \textit{reduce phase} in which a calculation (often an aggregation) is performed on them to generate the a single output value. The main benefit of this framework is that the data can be distributed across a large amount of machines (which are using for example GFS or HDFS). Additionally, instead of communicating the data between the nodes, the program is communicated between the nodes which is magnitudes smaller and more efficient to pass around. A general overview of the execution of a map-reduce problem is given in figure \ref{mapreduce_execution}.

\begin{figure}
  \includegraphics[width=\textwidth]{mapreduce_execution.png}
  \caption{Overview of the execution of a mapreduce problem\cite{Dean04}}
  \label{mapreduce_execution}
\end{figure}

Furthermore the MapReduce framework is similar to the \textit{Bulk-Synchronous processing (BSP)} style which is a little older. However, there are some differences as the MapReduce framework does not allow communication between nodes in the map phase, but only allows communication during the shuffle phase, in-between the mapping and reduce phase\cite{Pace12}.\\
Since BSP and Map-reduce are so similar, peope have worked on transforming BSP tasks into MapReduce tasks. Goodrich et al.\cite{Goo11} has shown that all BSP programs can be converted into MapReduce programs and other researchers have gone as far as to say that all MapReduce task are so similar to BSP tasks that, because BSP has a more theoretical basis, all tasks should be modeled as BSP tasks but implemented using the Map-reduce framework in order to gain the speed of MapReduce and the correctness of BSP\cite{Pace12}.

\paragraph{Apache Spark}
Like MapReduce, Apache Spark is another service built on top of a distributed file system to run programs on a distributed dataset. Spark is an open-source cluster-computing framework that is capable of executing an entire directed acyclic graph of transformations (like mappings) and actions (like reductions) fully in memory\cite{Sparkwebsite}. This is contrast to MapReduce, that forces the programmer to first use a mapping phase and then a reduce phase. This way, Spark is a lot faster than MapReduce because when, for example, 2 mapping phases are needed, 2 MapReduce tasks need to be executed which both require to write all (intermediate) data to the disk. Spark on the other hand can keep all the data in-memory which saves expensive writes to the disk, but needs to take additional steps to prevent losing processed data when there's a power outage.\\

The way in which Spark solves this problem is by using \textbf{RDDs} (Resilient Distributed Datasets). These datasets are read-only and new ones can only be created from data stored on the disk, or transforming existing RDDs\cite{Zaha12}. The Resilient part comes into play when the data is lost. Each RDD has a lineage graph, which shows what transformations have been executed on it. This means that once some data is lost, Spark can trace the path that RDD has followed by using the lineage graph and recalculate any lost data. It is important that the lineage graph does not contain any cycles, i.e. is a Directed Acylic Graph (DAG), because otherwise the data cannot be recovered as Spark will run into an infinite loop.

\begin{figure}
  \begin{center}
    \includegraphics[scale=0.5]{Lineage_graph.png}
  \end{center}
  \caption{Example of the lineage graph of an RDD\cite{Zaha12}}
  \label{lineagegraph}
\end{figure}

\subsubsection{Domain-specific implementations}

With the rising popularity of applied machine learning in many industries, a variety of domain-specific frameworks have been developed that have distribution models tailored towards ML. In this section, we analyze properties of the most popular implementations.

\paragraph{Distributed ensemble learning}
Many frameworks already exist for the development of machine learning models. However, they often have limited support for distributed training, even though they are fast and effective when used on a single machine or high-bandwidth cluster. One way to achieve distribution with these frameworks is through training separate model instances for disjoint subsets of the available data. At prediction time, the outputs of those instances can then be combined through standard ensemble model aggregation\citep{Opitz1999}.
Implementations that follow this strategy are not dependent on any specific library. They lend themselves well to implementation on top of standard distributed systems frameworks. The training process involves executing the individual model trainings on a number of independent machines in parallel. Neither orchestration nor communication are necessary once training has started. Training on $m$ machines with $m$ disjoint subsets of the dataset results in $m$ different models that are serialized somewhere. Each of these can have separate parameters, hyperparameters, and layer architecture. At prediction time, the trained model instances can then all be deserialized and run on new data. This can once again happen in a distributed fashion, either through data- or model-parallelism.
One large drawback is that this method is dependent on proper subdivision of the training data. If large biases are present in the training sets of some of the models, those instances could have a negative overall effect at prediction time.
There's a large number of frameworks available for this method, as any machine learning framework is suitable for it. Some popular implementations are Tensorflow\citep{Tensorflow2015}, MXNet\citep{MXNet2017} and PyTorch\citep{PyTorch2017}. Further analysis of these frameworks is out of scope.

\paragraph{Parallel synchronous SGD}


\subparagraph{Tensorflow \& Baidu AllReduce \citep{BaiduAllReduce2017}}

applies technologies as known from generic high performance compute (namely Message Passing Interface (MPI) and its AllReduce operation) to iteratively run model training on separate minibatches of the training data. AllReduce is used to, after each operation, apply each of the workers’ gradients onto the last common model state, and then propagate the result of that operation back to each worker. This is an inherently synchronous process, blocking on the result of each workers’ training iteration before continuing to the next.
Baidu includes a further optimization from \citet{Patarasuk2009} in this process called a Ring AllReduce , which is used to reduce the total amount of communication that is required to execute the full AllReduce operation. Whereas the naive approach to AllReduce (sending all workers’ state to a central node, which executes the reduction and sends back the result to each worker) would scale in bandwidth linearly with the number of nodes, the Ring AllReduce approach would avoid this by connecting each node only with one neighbor for receiving data, and one neighbor for sending data to. Each node would then execute a reduction on a subset of the reduced data array and forward the result of that to the next node, as long as necessary until each subset of the array has a fully reduced result available on one node. The results are then cycled through the ring step by step, until each node has each subset of the result available to it. Each comms operation can happen in parallel (utilizing all available bandwidth and putting the bottleneck of communication time at the latency of the slowest link between neighbor nodes) without risk of contention and related efficiency losses.
Baidu claims linear speedup when applying this technique to train deep learning networks. However, it has only been demonstrated on relatively small clusters (5 nodes each, though each node has multiple GPUs that communicate with each other through the same system). Also, although not explicitly mentioned, it is likely that their experimental setup makes use of an Infiniband high bandwidth link between each node, as opposed to the commodity machine approach that many other frameworks offer. The approach also lacks fault tolerance by default, as each node in the ring cannot be missed. This could be counteracted using redundancy, at cost of efficiency.
The system has been integrated into Tensorflow as an alternative to its parameter server-based approach that is described below.

\subparagraph{Caffe2}

(primarily maintained by Facebook) distributes ML (data-parallel) through, once again, AllReduce. It does this using NCCL between GPUs on a single host, and custom code between hosts that uses the Gloo[] library . Facebook uses Ring AllReduce (which offers better bandwidth \& parallelism guarantees) but also recursive halving and doubling (a divide-and-conquer approach) that offers better latency guarantees. According to their paper, this offers improved performance in latency-limited situations such as for small buffer sizes and large server counts. The end result of Facebook’s approach is that they can train ImageNet in the span of 1 hour\citep{Goyal2017}, achieving linear scaling with the number of GPUs. However, no fault-tolerance is present (much like in Baidu's approach).

\subparagraph{Horovod\citep{Horovod2018}}

takes a very similar approach to that of Baidu and its custom Tensorflow MPI module: it adds a layer of AllReduce-based MPI training to Tensorflow. One difference is that Horovod uses the NVIDIA Collective Communications Library (NCCL) for increased efficiency when training on (Nvidia) GPUs. Data-parallelizing an existing Tensorflow model is relatively simple (a few lines of code need to be added), after which the model can be optimized through synchronous SGD in the same Ring AllReduce-based manner as proposed by Baidu. However, Horovod lacks fault tolerance.

\subparagraph{CNTK}

offers multiple modes of data-parallel distribution, multiple of which (the synchronous ones) use the Ring AllReduce tactics previously described (again scaling linearly but lacking fault tolerance). Two innovations are offered by the library:

1-bit stochastic gradient descent (\citet{Seide2014}) is an implementation of SGD that quantizes training gradients to a single bit per value. This reduces the number of bits that need to be communicated when doing distributed training by a large constant factor.

Block-momentum SGD (\citet{Chen2016}) divides the training set into m blocks and n splits. Each of n machines trains a split in each block. The gradients calculated for all splits within a block are averaged to arrive at the weights for the block. The block updates are then merged into the global model, while applying block-level momentum and learning rate. 



\paragraph{Parallel stale-synchronous SGD}

\subparagraph{Petuum \citep{Xing2013}}

aims to provide a generic platform for any type of machine learning (as long as it is iteratively convergent) on big data and big models (terabytes/petabytes; hundreds of billions of parameters). It supports data- and model-parallelism. That approach exploits ML’s error tolerance, dynamic structural dependencies, and non-uniform convergence in order to achieve good scalability on large datasets and models. This is in contrast to e.g. Spark (focusing on fault tolerance \& recovery) and GraphLab (focusing on consistency), which can still be important to ML. The platform uses stale synchronicity (synchronous, but allowing bounded staleness) to exploit error tolerance (since a bit of staleness will have minor effects on convergence), dynamic scheduling policies to exploit dynamic structural dependencies (which helps minimize parallelization error and synchronization cost) and unconverged parameter prioritization to take advantage of non-uniform convergence (reducing computational cost on parameters that are already near optimal). 

Petuum uses the parameter server model (as described in the DistBelief section) to keep track of the parameters of the model being trained. The parameter server is also responsible for maintaining the staleness guarantees. In addition, it also exposes a scheduler that lets the model developer control the ordering of parallelized model updates.

The programming model that Petuum used is different to that of DistBelief in one core way: data- and model-parallelism operate at the same tier. Whereas in DistBelief model-parallelism is managed inside model shards, and each model replica operates on the parameter server as a single entity, Petuum instead has both model and data slices interact with the parameter server directly, and requires the user to define some central update logic if model-parallelism is involved. This allows slightly more programmer control.

When developing a model using Petuum, developers have to implement at least one method, named push, which is responsible for each of the parallelized model training operations. Its implementation should pull the model state from the parameter server, run a training iteration, and push a gradient to the parameter server. Petuum by default automatically manages the scheduling aspect and the parameter merging logic, so that data-parallel models don’t require any additional operations. If you want model-parallelism, however, you need to also implement the schedule method (which tells each of the parallel workers which parameters to train) and the pull method (which defines the aggregation logic for each of the parallelly generated parameter gradients).

Petuum provides an abstraction layer that also allows it to run on systems using YARN and HDFS, which simplifies compatibility with pre-existing clusters. 

\paragraph{Parallel asynchronous SGD \& Parameter Servers}

\subparagraph{DistBelief \citep{DistBelief2012}}

is one of the early practical implementations of large-scale distributed learning, initially developed by Google. They encountered the limitations of GPU training, and built DistBelief to counteract it. DistBelief supports data- and model-parallel training on tens of thousands of CPU cores. At the time of writing, it was used to successfully train models 30x larger than reported in preceding literature.

DistBelief imposes no restrictions on the structure of the models. 
At the time, Google also evaluated solutions based on MapReduce\citep{MapReduce} and GraphLab\citep{GraphLab}. MapReduce ”was ill-suited for the iterative computations inherent in deep network training”. GraphLab “would not exploit computing efficiencies available in the structured graphs typically found in deep networks”. This led the company to develop a domain-specific alternative instead.

To achieve efficient model-parallelism, DistBelief exploits the graphical nature of neural networks. Machines each execute the training of a part of the model, which can span subsets of multiple layers. Communication is only required at those points where a node’s output is used as the input of a node trained by another machine. To define a DistBelief model, “[t]he user defines the computation that takes place at each node in each layer of the model, and the messages that should be passed during the upward and downward phases of computation.” Partitioning of the model across a cluster is transparent and requires no structural modifications. Efficiency of a given partitioning, however, depends on the model’s connectivity structure and computational needs, and requires careful design. Locally connected models, for example, lend themselves for model parallelism, because of limited cross-partition communication. Fully connected models, on the other hand, have more substantial cross-partition dependencies and are therefore harder to efficiently distribute through DistBelief.

To further parallelize model training, data parallelism is applied on top of the model parallelism. A centralized sharded parameter server as proposed by \citet{Li2014Comms}\citep{Li2014Scaling} is used to allow each of a set of model replicas (which may internally be model-parallel) to share parameters. DistBelief supports two methods of data parallelism, both of which are resilient to processing speed variance between model replicas, as well as full replica failure. The first method is downpour SGD, an asynchronous alternative to the inherently sequential SGD. Each replica of the model fetches the latest model parameters from the parameter server every $n_{fetch}$ steps, updates these parameters in accordance with the model, and pushes the tracked parameter gradients to the parameter server every $n_{push}$ steps. Simple implementations would use $n_{fetch} = n_{push} = 1$, causing it to fetch the parameters before every training iteration, and push the iteration’s gradient once it is available. Communication overhead can be reduced by increasing either or both parameters. Fetches and pushes could also each be executed on a separate thread, which would require only weak synchronization between each other and the training thread.

DistBelief layers are developed in C++, 

Downpour SGD is resilient to machine failures more than SGD, as it allows training to continue even if some model replicas are offline. The optimization process itself, however, becomes less predictable due to parameters that are out of sync on the model replicas or between shards of the parameter server. No theoretical guarantees or citations are offered to support the robustness of this approach, yet the authors “found relaxing consistency requirements to be remarkably effective.” Tactics that contribute to robustness are the application of adaptive learning rates through AdaGrad[cite adagrad here] and “warmstarting “the model through training a single model replica for a while before scaling up to the full number of machines. The authors make note of not having stability issues after applying these.
The second method of parallelization is distributed L-BGFS. This makes use of an external coordinator process that divides training work between model replicas, as well as some operations on the parameters between the parameter server shards.

The shards of the parameter server each hold a fraction of the total parameter space of the model being trained. The model replicas pull the parameters from all shards; each parallelized part of the model only retrieves those parameters that it needs.

Performance improvements are high but very expensive in terms of CPU hours. DistBelief also did not support distributed GPU training at the time of \citet{DistBelief2012}, which would probably reduce training time quite significantly.

\subparagraph{Tensorflow \citep{Tensorflow2015}\citep{Tensorflow2016}}

is the evolution of DistBelief, in the sense that it was developed to replace DistBelief within Google, and borrows both concepts and experience from it. By "both simplifying and generalizing [DistBelief]"\citep{Tensorflow2016}, the Tensorflow developers intend to cater to a wider variety of ideas.

TensorFlow represents both model algorithms and state as a dataflow graph, execution of which can be distributed. This facilitates different parallelization schemes that can take e.g. state locality into account. The level of abstraction of the dataflow graph is that of mathematical operations on tensors (i.e. $n$-dimensional matrices), as opposed to DistBelief, which abstracted at the layer level. Consequently, defining a new type of neural network layer in Tensorflow requires no custom code - it can just be a subgraph composed from simpler math operations.

A Tensorflow model is first defined as a symbolic dataflow graph. Once this is done, the graph is optimized and then executed on the available hardware. One advantage of this model is that it allows Tensorflow to tailor its execution strategy towards the types of devices available to it. When working with e.g. GPUs or TPUs (Tensor Processing Units\citep{TPU2017})


\paragraph{Parallel hybrid-synchronous SGD}

Both synchronous and asynchronous approaches have some significant drawbacks, as is explored by \citet{ChenJianmin2016}. A few frameworks attempt to find a middle ground instead that combines some of the best properties of each model of parallelism, and diminishes some of the drawbacks.

\subparagraph{MXNet-MPI \citep{Mamidala2018}}

takes an approach to distributed ML (using a modified version of MXNet as a proof of concept) that combines some of the best aspects of both asynchronous (parameter server) and synchronous (MPI) implementations. The idea here is to use the same approach as described in the MXNet section, but instead of having single workers communicate with the parameter server, to cluster those workers together into groups that internally apply synchronous SGD over MPI/AllReduce. This has the benefits of easy linear scalability of the sync MPI approach, and fault tolerance of the async PS approach.



\subsubsection{Current challenges}
\paragraph{Performance}

A trade-off that’s seen frequently is the reduction of wall-clock time at the expense of total processing time (i.e. decreased efficiency). When compute resources are affordable enough, many real-world use cases of machine learning benefit most from being trained rapidly. The fact that this often implies a large multiple in total compute resources (and the associated energy consumption) is not as important, as long as a model saves more money than it costs to run.  A good example of this is found in \citet{DistBelief2012}, where wall clock time speedup factors are achieved by increasing the number of machines quadratically or worse. It still delivered Google competitive advantage for years. Distributed use of GPUs, as in Tensorflow, has better properties, but often still exhibits efficiency below 75\%.
These performance concerns are much less severe in the context of synchronous SGD-based frameworks, which often do achieve linear speedups in benchmarks. However, most of these benchmarks test at most a few hundred machines, whereas the scale at which e.g. DistBelief is demonstrated can sometimes be two orders of magnitude larger. The reason for this is unclear, although it might be related to the fault tolerance concerns mentioned below. Synchronous frameworks also benefit more from high-bandwidth links such as InfiniBand due to the fact that all nodes' communication is synchronized, even though they are expensive (with costs running up to thousands of euros for even a cheap switch) and can't practically be used in large footprint clusters (e.g. in multiple datacenters) due to wiring and latency constraints.

\paragraph{Fault tolerance}

Synchronous AllReduce-based approaches seem to scale significantly better than the parameter server approach (up to a certain cluster size), but suffer from a lack of fault-tolerance: failure of a single machine blocks the entire training process. At smaller scales this might still be a manageable problem, but past a certain number of nodes the probability of any node being unavailable becomes high enough to lead to near-continuous stalling. Common implementations of these HPC-inspired patterns, such as MPI and NCCL, lack fault-tolerance completely, and although there are efforts to counteract some of this, production-ready solutions are lacking. Some of the described implementations allow for checkpointing to counteract this, but a lot of work is necessary to enable true fault-tolerance, such as described in \citet{Amatya2017}. It is also possible to reduce the probability of failure for each individual node, but this requires very specific hardware that is expensive (e.g. highly stable datacenter cooling or Infiniband networks) and not generally available on commodity cloud platforms.
Asynchronous implementations do not suffer from this problem quite as much, as they are designed to explicitly tolerate straggling and failing nodes, with only minimal impact on training performance. The question for ML practitioners, then, is whether they prefer performance or fault tolerance, and whether they are constrained by either one. Hybrid approaches even offer a way to customize these characteristics, although they are not frequently found in use yet. It would be interesting to see whether an even better approach exists, or whether there is an efficient way to implement fault-tolerant AllReduce.

\paragraph{Privacy}

There are scenarios in which it is beneficial or even mandatory to isolate different subsets of the training data from each other. The furthest extent of this is when a model needs to be trained on datasets that each live on different machines/clusters, and may under no circumstance be colocated or even moved.
One interesting approach to training a model in a privacy-sensitive context is the use of a distributed ensemble model. This allows perfect separation of the training data subsets, with the drawback that a method needs to be found to properly balance each trained model's output for an unbiased result.
Parameter server-based systems could also be useful in the context of privacy, as the training of a model can be separated from the training's result. However, this assumes that no sensitive properties of the underlying data leak into the model itself, which is not easy to prove.
Finally, it is possible to introduce statistical noise into each subset of the training data, with the intention of rendering its sensitive characteristics unidentifiable to other parties. \citet{Bal12} touches on this subject, but makes it clear that the amount privacy in this scenario is dependent on the amount of statistical queries required to learn the dataset, which puts an upper bound on the usefulness of the model itself.
In conclusion, it is highly unlikely that perfect privacy is possible, but current frameworks do not offer much support for even basic variants. It could be interesting to answer whether there's a generic way to facilitate distributed privacy, which could then be integrated into the frameworks that are being used.

