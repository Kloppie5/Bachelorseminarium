\section{Results}
% Answer research questions

\subsection{The need for Distributed Machine Learning}
\subsubsection{Alternatives to Distributed Machine Learning}
\paragraph{Long-term sustainability}










\subsection{Underlying technology}
To give you an overview of how Distributed Machine Learning works, we'll give you an abstract framework that includes everything a real implementation should include. We do that by exploiting the three unique properties of Distributed Machine Learning, namely error tolerance, dynamic structural dependency and non-uniform convergence.\cite{Xing16}\\
Our goals include (1) to list regular Machine Learning algorithms that are commonly used in a distributed setting; (2) to find algorithms to determine the best parameters for the former algorithms; (3) to trade-off computation time with communication and accuracy; and (4) to minimize the amount of bits sent over the network so that the system is no bottlenecked by scarce network bandwidth.\\
Designing a general system in such a way that the regular Machine Learning algorithms can be distributed efficiently is challenging, because every algorithm has its own communication patterns \cite{Jia14}\cite{Newman09}\cite{Rich13}\cite{Smola10}\cite{Takac13}\cite{Tsi12}.

\subsubsection{Machine Learning algorithms}
Machine learning algorithms learn to make predictions or decisions based on data. Every machine learning algorithm has its own pros and cons. Machine learning algorithms can be divided into categories based on some of their characteristics;
\begin{itemize}
	\item \textbf{Feedback}, the type of feedback given to the algorithm during its learning
	\item \textbf{Goal}, the type of output
	\item \textbf{Type}, the way they create their output
	\item \textbf{Method}, the way they change when given feedback
\end{itemize}

\paragraph{Feedback}
\begin{itemize}
	\item \textbf{Supervised}
		learning uses training data which consists of input objects, usually vectors, and output values. Supervised learning algorithms typically try to find a function to map input to output and use that to find outputs for unknown input objects.
		% bias vs variance
		% complexity vs amount of data
		% dimensionality
		% noise
	\item \textbf{Unsupervised}
		learning uses training data without labels, which means there is no way to look at the accuracy of the output. It is more commonly used to find clusters of datapoints or patterns.
	\item \textbf{Semi-supervised}
		learning make use of a usually small amount of labeled data and a large amount of unlabeled data. Clustering can be used to extend the labels known to other datapoints. This is done under the assumption that similar datapoints share a label.
	\item \textbf{Reinforcement learning}
		is done by having the model generate an output and having a different system give feedback on how well the output matches the desired output. This is done by either minimizing a risk function or by maximizing a reward function. This means that reinforcement learning can be done without having a big dataset of correct and incorrect input-output pairs. Because of the black box behavior of the model, it is not possible to explicitly correct actions that are sub-optimal and the model might get stuck in a local minimum.
		% Q-learning
\end{itemize}

\paragraph{Goals}
\begin{itemize}
	\item \textbf{Anomaly detection}
		can be divided into three categories. \textbf{Supervised anomaly detection} requires data that has a different label for abnormal data and trains a normal classifier. \textbf{Unsupervised anomaly detection} assumes that normal and abnormal datapoints will be groups separately and effectively does clustering. \textbf{Semi-supervised anomaly detection} builds a model of normal data and test whether new datapoints are normal or abnormal. Based on the dataset, different methods perform better than others
	\item \textbf{Classification}
		is the problem of putting new datapoints in categories based on the categories of the training data. This is an inherently supervised process. The unsupervised version of this is Clustering.
	\item \textbf{Clustering}
		is the problem of grouping together datapoints that are similar according to some criteria. This can be done supervised, but is usually done with significantly big datasets for which labeling might be too expensive
	\item \textbf{Dimensional reduction}
		is the problem of reducing the amount of variables looked. This can be done by either selecting only those variables that are relevant, called \textbf{Feature selection}, or by creating new variables that represent multiple other variables, called \textbf{Feature extraction}.
	\item \textbf{Feature learning / Representation learning}
		is the set of techniques designed for finding good representations of the data for things like feature detection, classification, clustering, encoding and even matrix factorization.
		% manifold learning
		% statistical inference/density estimation
		% sparse coding
	\item \textbf{Regression}
		is the problem of estimating how a dependent variable changes with changes to the independent variables. Parametric regression tries to find the parameters of a function, which speeds up the process if for example a linear correlation is expected. Nonparametric regression also tries to find the type of function, but this requires a larger sample size and significantly more time.
\end{itemize}

\paragraph{Types}
\begin{itemize}
	\item \textbf{Evolutionary algorithms (EAs)},
		specifically \textbf{Genetic algorithms (GAs)} are algorithms that learn iteratively based on evolution. The algorithm that actually solves the problem is represented by a set of data that determine its properties, called its \textbf{genotype}. The performance of the algorithm is measured using an objective score, calculated using a \textbf{fitness function}. After calculating the fitness of all generated algorithms, the next iteration creates new genotypes based on mutation and crossover of algorithms that are 'more fit'. Generic algorithms can be used to create other algorithms like neural networks, belief networks, decision trees and rule sets.
	% Graphical models
	\item \textbf{Perceptron-based}
		algorithms are based on perceptrons, binary classifiers that map an input vector to being 'active' or 'inactive' by assigning a weight to all inputs and summing over the products of these weights and their inputs and comparing them to a threshold number, usually called the bias. Perceptron-based algorithms commonly use the entire batch of training data to try to find an instance that will work for the entire set. Perceptron-based algorithms are binary and are therefore mainly used binary or multi-label classification.
	\item \textbf{Rule-based machine learning (RBML)}
		algorithms use a set of rules that each represent a small part of the problem. These rules usually express a condition and a result for when that condition is met. Because of this if-then relation, rules lend themselves to be more easily interpreted than more abstract types of machine learning algorithms such as neural networks.
	\item \textbf{Topic Models (TM)}
		are statistical models for finding and mapping semantic structures in text.
\end{itemize}

\paragraph{Methods}
\begin{itemize}
	\item \textbf{Association rule learning}
		is a \textit{rule-based machine learning} method that focuses on finding relations between different variables in datasets. This is done by looking at some measure of interest. Examples of this are \textbf{Support}, how often variables appear together; \textbf{Confidence}, how often a causal rule is true; and \textbf{Collective Strength}, a comparison of the amount of instances which contains some but not all variables in the relation and the expected amount if the variables where not related.
	\item \textbf{Artificial neural networks (ANNs)}
		are perceptron-based systems using multiple layers. These layers are usually divided into an input layer, an output layer, and one or more hidden layers. Each layer consists of nodes connected to the previous and next layers by weights, usually called synapses. Unlike normal perceptrons, nodes usually use an activation function instead of just a bias.
		The workings of the algorithm is dependent on the entire network, the algorithm can be changed by changing (1) the weights of the synapses, (2) the layout of the network or (3) the activation function of nodes.
		Because neural networks require a big amount of nodes, the un-understandability of what a neural network actually does is a major drawback when comparing it to for example decision trees.
		Neural networks are extensively studied for their ability to analyze enormous sets of data. Neural networks can be divided into subgroups based on the layout of the network;
		\begin{itemize}
			\item \textbf{Recurrent neural networks (RNNs)}
				have synapses going back to previous layers, which means that the previous state of the network influences its current decisions. Neural networks that are not recurrent are called \textbf{feed-forward}. Recurrent synapses give the network a sort of memory that can help with discovering patterns in data that arrives sequentially. Special blocks of nodes in a network can work as a memory cell that can hold some information for an arbitrarily long timespan. These blocks are called long short-term memory (LSTM) units.
				% finite impulse
				% 	DAG which can become a feedforward network 
				% vs infinite impulse
				% 	DAG which can not be unrolled
			\item \textbf{Hopfield networks}
				are a type of non-reflexive, symmetric recurrent neural network that have some 'energy' related to every state of the network as a whole which will reach a local minimum after continuously updating the network.
			\item \textbf{Deep neural networks (DNNs)},
				opposite of \textbf{shallow neural networks}, have many hidden layers, which may cause the network to work as a sort of black box
			\item \textbf{Convolutional neural networkss (CNNs / ConvNets)}
				are deep, feed-forward neural networks that use convolution layers with nodes connected to only a few nodes in the previous layer. These values are then pooled using pooling layers that can be seen as a way of recognizing abstract features in the data. The convolution makes the network look at local data, making the entire algorithm spatially invariant, which is why they are sometimes called space invariant artificial neural networks (SIANN). Chaining multiple of these convolution and pooling layers together can make the network very good at recognizing complicated constructs in big datasets, like recognizing cars in images or the contextual meaning of a sentence in a paragraph.
			\item \textbf{Generative adversarial networks (GANs)}
				consist of two separate networks, one trying to recognize objects from a dataset and one trying to create new data in an attempt to 'fool' the other network into thinking the data is legit.\cite{Li:2013:CAL:2463372.2465801}
			\item \textbf{Radial Basis Function (RBF) networks}
				\cite{rbf}
			\item \textbf{Self-organizing maps (SOMs) / self-organizing feature maps (SOFMs)}
				are neural networks that learn by unsupervised \textbf{competitive learning}, in which nodes compete for access to specific inputs, causing the nodes to become highly specialized, which reduces redundancy. Iterations effectively move the map closer to the training data, which is where it gets its name from. Some subtypes include time adaptive self-organizing map (TASOM), Binary Tree TASOM (BTASOM) and growing self-organizing map (GSOM)
			\item \textbf{Stochastic neural networks}
				make use of stochastic transfer functions or weights, which makes it possible to escape a local minimum. 
			% 		boltzmann machine
		\end{itemize}
		Neural networks trained in many different ways, like using Generic algorithms. The most common approach, especially when talking about Deep neural networks, is using a process called \textbf{backpropagation};
		\begin{itemize}
			\item Present a training sample
			\item Calculate the error in each output neuron, how much lower or higher the output must be adjusted to match the desired output
			\item Calculate the gradient
			\item For every layer, adjust the weights and biases based on the gradient
			\item Repeat
		\end{itemize}
	\item \textbf{Autoencoders}
		are a type of neural network that trained specifically to encode and decode data. Because autoencoders are trained to also decode, the encoded version of the data can be seen as a dimensional reduction of the data
	\item \textbf{Bayesian networks}, 
		sometimes called \textbf{belief networks}, are probabilistic directed acyclic graphical models\cite{Wain08}\cite{Kol09}\cite{Xin16} that are used to represent conditional relationships between variables. They are directed to overcome the problems of \textbf{Markov random
		fields (MRFs)}, also called \textbf{Markov networks}, which use undirected connections. These undirected connections make it impossible to represent dependencies that are non-transitive or otherwise induced. Bayesian networks are commonly used to represent \textbf{Markov processes} (not related to Markov networks) in particular for probabilistic inference or parameter estimations.
		\textbf{Nonparametric Bayesian models}\cite{Grif05}\cite{Teh06} don't fix the parameters in place, allowing the model to grow with the data size.
	%		Regularized Bayesian models \cite{Zhu09}\cite{Zhu09-2}\cite{Zhu14}
	\item \textbf{Decision trees},
		sometimes called CART trees, after Classification And Regression Trees trees (RAS syndrome), use Rule-based machine learning to create a set of rules that create decision branches. Traversing the tree means applying the rules at each step till you arrive at a leaf of the tree, which represents the decision or classification for that input.
	\item \textbf{Latent Dirichlet Allocation}\cite{Blei03}
		makes a mapping between documents and probabilistic set of topics, using the assumption that documents have few different topics and topics use few different words.
	\item \textbf{Latent semantic analysis (LSA) / latent semantic indexing (LSI)}
		creates a big matrix of documents and topics in an attempt to classify documents or to find relations between topics. LSA/LSI assumes a Guassian distributes for topics and documents. LSA/LSI doesn't have a way of dealing with words that have multiple meanings.
	\item \textbf{Naive Bayes classifiers}
		are relatively simple probabilistic classifiers that assume different features are independent. They can be trained quickly using supervised learning, but are less accurate than more complicated approaches.
	\item \textbf{Probabilistic latent semantic analysis (PLSA) / probabilistic latent semantic indexing (PLSI)}
		is the same as LSA/LSI, except that PLSA/PLSI assumes a Poisson distributes for topics and documents instead of the Guassian distribution that is assumed by LSA/LSI, because the Poisson distribution appears to model the real world better. Some subtypes include Multinomial ASymmetric Hierarchical Analysis (MASHA), Hierarchical Probabilistic Latent Semantic Analysis (HPLSA) and Latent Dirichlet Allocation (LDA).
	% Inductive logic programming (ILP)
	\item \textbf{Support vector machines (SVMs)}
		map datapoints to high dimensional vectors for classification and clustering purposes. For datapoints in p-dimensional space, a (p-1)-dimensional hyperplane can be used as a classifier. A reasonable choice would be the hyperplane that properly separates the datapoints in two groups based on their labels, that also has the biggest possible margin. Sometimes special transformation equations, called \textbf{kernels}, are used to change all datapoints to a different representation in which it is easier to find such a hyperplane.
		Having a lot of dimensions decreases the accuracy of classifying new datapoints, so increasingly large datasets are needed to make the algorithms perform well
\end{itemize}

Usually a single algorithm isn't enough to solve a problem, and multiple algorithms are combined in something ofter called \textbf{Ensemble Learning}. There are many different ways to do this, depending on what kind of algorithms are used;

\begin{itemize}
	\item \textbf{Bagging}
		is the process of building multiple classifiers and combining them into one.
	\item \textbf{Boosting}
		is the process of training new models with the data that is misclassified by the previous models.
	\item \textbf{Bucketing}
		is the process of training many different models and eventually selecting the one that has the best performance.
	\item \textbf{Random Forests}\cite{Breiman2001}
		use multiple decision trees and averaging the prediction made by the individual trees as to increase the overall accuracy. Different trees are given the same 'voting power'.
	\item \textbf{Stacking}
		is when multiple classifiers are trained on the dataset and one new classifier uses the output of the other classifiers as input in an attempt to reduce the variance.
	\item \textbf{Learning Classifier Systems (LCSs)}
		(not to be confused with the more common usage of the LCS initialism, Longest Common Subsequence) is a kind of modular system of learning approaches. An LCS iterates over datapoints from the dataset and completes the entire learning process in each iteration. The main idea is that an LCS has a limited number of rules that are follow a Genetic Algorithm that forces rules that are suboptimal out of the ruleset. There are many different overall attributes that can drastically change the performance of the LCS based on the dataset;
		\begin{itemize}
			\item Michigan-style architecture vs Pittsburgh-style architecture
			\item supervised vs reinforcement
			\item incremental vs batch
			\item online vs offline
			\item strength-based vs accuracy-based
			\item complete mapping vs best mapping
		\end{itemize}
		Because of increasing complexity of the training of LCS with respect to the size of the data, LCS is mainly used 
\end{itemize}


%Deep learning
%
%Multi-task learning (MTL) 
%
%Matrix factorization
%Singular value decomposition (SVD) and principal component analysis (PCA)


\paragraph{Algorithms to find best parameters}
To find the parameters for these algorithms we can use several Machine Learning workhorse implementations that can be re-used across different Machine Learning algorithm-families. These include:
\begin{itemize}
	\item First-order techniques
	\begin{itemize}
		\item Stochastic gradient descent
% 			Backpropagation
		\item Stochastic dual coordinate ascent\cite{Shal13}
		\item L-BFGS
		\item Conjugate gradient
	\end{itemize}
	\item Second-order techniques
	\begin{itemize}
		\item Newton descent
		\item Quasi-Newton descent
	\end{itemize}
	\item Coordinate descent
	\item Markov-Chain Monte-Carlo
	\item Variational inference
\end{itemize}


% --------


\subsubsection{Partitioning and distribution algorithms}
Now we'll look the abstraction of the general design for a distributed Machine Learning operating system that executes the Machine Learning workhorses across a wide variety of hardware.


\paragraph{Computation time vs communication vs accuracy}


\paragraph{Scheduling and balancing the workloads}
There are 3 things to take into account when partitioning an ML program in order to parallelize it:\cite{Xing16}\\
\begin{enumerate}
	\item Deciding which tasks go or don' go together in parallel
	\item Deciding the order in which tasks will be executed
	\item Ensuring an even load distribution across the machines
\end{enumerate}


\paragraph{Bridging computation and communication}
Let's look at several ways to efficiently communicate between the nodes taking the interleaving of parallel program computations and inter-worker communication into account.
\begin{enumerate}
	\item \underline{BSP:} Bulk Synchronous Parallel, the most simple model; programs will alternate between a computation and a communication phase to ensure consistency\cite{Xing16}. An example of program following the BSP bridging model is MapReduce.\\
	An advantage is that serializable BSP ML programs are guaranteed to output a correct solution. A disadvantage is that finished workers must wait at every synchronization barrier till the other works are finished, which results in overhead\cite{Chilimbi14}. Another disadvantage is that the synchronization barrier may take a significant amount of time, because of slow communication between the workers.
	\item \underline{SSP:} Stale Synchronous Parallel, allows the fastest worker to be ahead of the slowest worker up to a bounded number of iterations before the workers are stopped. The fast workers may work for a while on state data, but SSP still limits the maximum staleness between any pair of workers. An advantage is that it still enjoys strong model convergence guarantees. A disadvantage is that, when machines temporarily slow down due to other tasks or users and cause large staleness, the convergence rates are poor.
	\item \underline{ASP:} Approximate Synchronous Parallel, limits how inaccurate a parameter can be, in contrast to SSP that limits how stale a parameter can be. An advantage is that, when an aggregated update is insignificant, then the server can delay synchronization indefinitely. A disadvantage is that it can be hard to choose the parameter that defines which update are significant and which are not. \cite{Hsieh17}
	\item \underline{BAP / TAP:} Barrierless Asynchronous Parallel\cite{Han15} or Total Asynchronous Parallel\cite{Hsieh17}; worker machines communicate in parallel without waiting for each other. The advantage is that it usually obtains the highest speedup possible. A disadvantage is that the model converges slowly or even incorrectly because the error grows with the delay, unlike BSP and SSP. \cite{Han15}
\end{enumerate}
At the moment, ASP is the state-of-the-art method.


\paragraph{Communication strategies}
To distribute Machine Learning algorithms we can choose to partition either the data or the model across the machines - referred to respectively as data parallelism and model parallelism \cite{Die12}. These two types of parallelism can be applied at the same time \cite{Xing16}. Data parallelism can be applied to every ML algorithm with an i.i.d. assumption over the data samples, which are most ML algorithms \cite{Xing16}. It partitions the data and assigns it to parallel machines. Model parallelism cannot be applied to most ML algorithms, because the model parameters generally don't have this i.i.d. assumption. It partitions the model parameters and assigns that to parallel machines.\\
There are several communication management strategies\cite{Xing16} used to spread and reduce the amount of data communicated between machines:
\begin{itemize}
	\item To prevent bursts of communication over the network, for example after a mapper is finished, continuous communication is used, for example in the state-of-the-art implementation Bösen\cite{Wei15}.
	
	\item Neural networks are composed out of layers, of which the training by using the back-propagation gradient descent algorithm is highly sequential. Because the top layers of neural networks contain a lot more parameters while it accounts only for a small part of the total computation \cite{Xing16}, WFBP \cite{Zhang17} was proposed to spread the computation and communication out in an optimal fashion.
	
	\item Hybrid communication (HybComm) Because WBFP does not reduce the communication overhead, HybComm \cite{Zhang17} was proposed. Effectively it combines Parameter Servers (PS)\cite{Wei15} with Sufficient Factor Broadcasting (SFB)\cite{Xie15} by being aware of both the mathematical property of neural networks and the structure of computing clusters. See below for more information about PS and SFB.
	
\end{itemize}


\paragraph{Network topologies}
There are several network topologies used for Distributed Machine Learning clusters:
\begin{itemize}
	\item \textbf{Trees} AllReduce\cite{Agar14}
	\item \textbf{Centralized storage} Parameter Server (PS)\cite{Agar14}
	\item \textbf{Decentralized storage} e.g. peer-to-peer storage (e.g. Sufficient Factor Broadcasting (SFB)\cite{Li13})
\end{itemize}










\subsection{Currently used implementations}
\subsubsection{Generic distributed system frameworks}
With the rising popularity of applied machine learning in many industries, a variety of domain-specific frameworks have been developed that have distribution models tailored towards ML. In this section, we summarize the characteristics of the most popular implementations.

\paragraph{Distributed ensemble learning}
Many frameworks already exist for the development of machine learning models. However, they often have limited support for distributed training, even though they are fast and effective when used on a single machine or high-bandwidth cluster. One way to achieve distribution with these frameworks is through training separate model instances for disjoint subsets of the available data. At prediction time, the outputs of those instances can then be combined through standard ensemble model aggregation\citep{Opitz1999}.
Models that follow this strategy are not dependent on any specific library. They lend themselves well to execution that uses standard distributed systems frameworks. The training process involves executing the individual model trainings on a number of independent machines in parallel. Neither orchestration nor communication are necessary once training has started. Training on $m$ machines with $m$ disjoint subsets of the dataset results in $m$ different models that are serialized somewhere. Each of these can have separate parameters, hyperparameters, and layer architecture. At prediction time, the trained model instances can then all be deserialized and run on new data. This can once again happen in a distributed fashion, through data- and/or model-parallelism.
One large drawback is that this method is dependent on proper subdivision of the training data. If large biases are present in the training sets of some of the models, those instances could have a negative overall effect at prediction time. If the data is divided manually, it's important to ensure independence and indentical distribution. If, on the other hand, the dataset is inherently divided and distributed, this is not quite as straightforward.
There's a large number of existing frameworks available for this method, as any machine learning framework can be used. Some popular implementations are Tensorflow\citep{Tensorflow2015}, MXNet\citep{MXNet2015} and PyTorch\citep{PyTorch2017}. Further analysis of these frameworks is out of scope.

\paragraph{Parallel synchronous SGD}


\subparagraph{Tensorflow \& Baidu AllReduce \citep{BaiduAllReduce2017}}

uses common high performance compute technology (mainly Message Passing Interface (MPI) and its AllReduce operation) to iteratively run SGD model training on separate minibatches of the training data. AllReduce is used to, after each operation, apply each of the workers’ gradients onto the last common model state, and then propagate the result of that operation back to each worker. This is an inherently synchronous process, blocking on the result of each workers’ training iteration before continuing to the next.
Baidu includes a further optimization from \citet{Patarasuk2009} in this process called a Ring AllReduce to reduce the required amount of communication for executing the AllReduce operation. Whereas the naive approach to AllReduce (sending all workers’ state to a central node, which executes the reduction and sends back the result to each worker) would scale in bandwidth linearly with the number of nodes, the Ring AllReduce approach would avoid this by connecting each node only with one neighbor for receiving data, and one neighbor for sending data to. Each node would then execute a reduction on a subset of the reduced data array and forward the result of that to the next node, as long as necessary until each subset of the array has a fully reduced result available on one node. The results are then cycled through the ring step by step, until each node has each subset of the result available to it. Each communication step can be parallellized (utilizing all available bandwidth and putting the bottleneck of communication time at the latency of the slowest link between neighbor nodes) without risk of contention and related efficiency losses.
Baidu claims linear speedup when applying this technique to train deep learning networks. However, it has only been demonstrated on relatively small clusters (5 nodes each, though each node has multiple GPUs that communicate with each other through the same system). Also, although not explicitly mentioned, it is likely that their experimental setup makes use of high bandwidth links between each node, as opposed to the commodity machine approach that many other frameworks offer. The approach lacks fault tolerance by default, as no node in the ring can be missed. This could be counteracted using redundancy (at cost of efficiency), but if not done, the scalability of this method is limited by the probability that every node is available. This can be high when using large numbers of commodity machines and networking to facilitate large datasets.
The system has been integrated into Tensorflow as an alternative to its parameter server-based approach (described below).

\subparagraph{Horovod\citep{Horovod2018}}

takes a very similar approach to that of Baidu and its custom Tensorflow MPI module: it adds a layer of AllReduce-based MPI training to Tensorflow. One difference is that Horovod uses the NVIDIA Collective Communications Library (NCCL) for increased efficiency when training on (Nvidia) GPUs. This also enables use of multiple GPUs on a single node. Data-parallelizing an existing Tensorflow model is relatively simple (a few lines of code need to be added), after which the model can be optimized through synchronous SGD in the same Ring AllReduce-based manner as proposed by Baidu. When benchmarked on Inception v3\citep{Szegedy2015} and ResNet-101\citep{He2015} using 128 GPUs, efficiency of distribution is about 88\%, compared to about 50\% in Tensorflow's parameter server approach.
However, Horovod lacks fault tolerance (much like in Baidu's approach), and the same scalability concerns therefore arise.

\subparagraph{Caffe2}

(primarily maintained by Facebook) distributes ML (data-parallel) through, once again, AllReduce. It does this using NCCL between GPUs on a single host, and custom code between hosts that uses Facebook's Gloo library, which abstracts over multiple transports (e.g. TCP/IP or InfiniBand). Facebook uses Ring AllReduce (which offers better bandwidth \& parallelism guarantees) but also recursive halving and doubling (a divide-and-conquer approach that offers better latency guarantees). According to their paper, performance is improved in latency-limited situations such as for small buffer sizes and large server counts. The end result of Facebook’s approach is that they can train ResNet-50\citep{He2015} in the span of 1 hour\citep{Goyal2017}, achieving linear scaling with the number of GPUs, at 90\% efficiency and measured up to 352 GPUs. However, once again no fault-tolerance is present, which raises some questions on the topic of scalability.

\subparagraph{CNTK}

offers multiple modes of data-parallel distribution, a subset of which (the synchronous ones) use the Ring AllReduce tactic previously described, with the same tradeoff (achieving linear scalability at expense of fault tolerance). Two innovations are offered by the library:

1-bit stochastic gradient descent (\citet{Seide2014}) is an implementation of SGD that quantizes training gradients to a single bit per value. This reduces the number of bits that need to be communicated when doing distributed training by a large constant factor.

Block-momentum SGD (\citet{Chen2016}) divides the training set into m blocks and n splits. Each of n machines trains a split in each block. The gradients calculated for all splits within a block are averaged to arrive at the weights for the block. The block updates are then merged into the global model, while applying block-level momentum and learning rate.

When benchmarked on a Microsoft speech LSTM, average speedups of 85\%+ are achieved for small numbers of GPUs (up to 16), but scalability drops significantly (below 70\%) when scaling past that \footnote{Direct comparison of this number to the other synchronous frameworks' results is somewhat unfair, as the dependency structure of an LSTM is significantly different than that of an ordinary DNN due to the introduction of temporal state.}.


\paragraph{Parallel asynchronous SGD \& Parameter Servers}

\subparagraph{DistBelief \citep{DistBelief2012}}

is one of the early practical implementations of large-scale distributed learning, developed by Google. They encountered the limitations of GPU training, and built DistBelief to counteract them. DistBelief supports data- and model-parallel training on tens of thousands of CPU cores (though GPU support was later introduced as well\citep{Tensorflow2016}). At the time of the paper's writing, it was used to successfully train models 30x larger than reported in preceding literature.
 
At the time, Google also evaluated solutions based on MapReduce\citep{MapReduce} and GraphLab\citep{GraphLab}. MapReduce ”was ill-suited for the iterative computations inherent in deep network training”. GraphLab “would not exploit computing efficiencies available in the structured graphs typically found in deep networks”. This led the company to develop a domain-specific alternative instead.

To achieve efficient model-parallelism, DistBelief exploits the graphical nature of neural networks. Machines each execute the training of a part of the model, which can span subsets of multiple layers. Communication is only required at those points where a node’s output is used as the input of a node trained by another machine. To define a DistBelief model, “[t]he user defines the chilimbiomputation that takes place at each node in each layer of the model, and the messages that should be passed during the upward and downward phases of computation.” Partitioning of the model across a cluster is transparent and requires no structural modifications. Efficiency of a given partitioning, however, depends on the model’s connectivity structure and computational needs, and requires careful design. Locally connected models, for example, lend themselves for model parallelism, because of limited cross-partition communication. Fully connected models, on the other hand, have more substantial cross-partition dependencies and are therefore harder to efficiently distribute through DistBelief.

To further parallelize model training, data parallelism is applied on top of the model parallelism. A centralized sharded parameter server is used to allow each of a set of model replicas (which may internally be model-parallel) to share parameters. DistBelief supports two methods of data parallelism, both of which are resilient to processing speed variance between model replicas, as well as full replica failure. The first method is downpour SGD, an asynchronous alternative to the inherently sequential SGD. Each replica of the model fetches the latest model parameters from the parameter server every $n_{fetch}$ steps, updates these parameters in accordance with the model, and pushes the tracked parameter gradients to the parameter server every $n_{push}$ steps. Simple implementations would use $n_{fetch} = n_{push} = 1$, causing it to fetch the parameters before every training iteration, and push the iteration’s gradient once it is available. Communication overhead can be reduced by increasing either or both parameters. Fetches and pushes could also each be executed on a separate thread, which would require only weak synchronization between each other and the training thread.

The downpour SGD algorithm primarily used in DistBelief is resilient to machine failures more than SGD, as it allows training to continue even if some model replicas are offline. The optimization process itself, however, becomes less predictable due to parameters that are out of sync on the model replicas or between shards of the parameter server. No theoretical guarantees or citations are offered to support the robustness of this approach, yet the authors “found relaxing consistency requirements to be remarkably effective.” Tactics that contribute to robustness are the application of adaptive learning rates through AdaGrad\citep{Duchi2011} and “warmstarting “the model through training a single model replica for a while before scaling up to the full number of machines. The authors make note of not having stability issues after applying these.
A second method of parallelization is distributed L-BGFS. This makes use of an external coordinator process that divides training work between model replicas, as well as some operations on the parameters between the parameter server shards.

The shards of the parameter server each hold a fraction of the total parameter space of the model being trained. The model replicas pull the parameters from all shards; each parallelized part of the model only retrieves those parameters that it needs.

Performance improvements are high but very expensive in terms of CPU hours: while the best speedup (downpour SGD with AdaGrad) achieved an 80\% decrease in training time on ImageNet, this was achieved using more than 500 machines and more than 1000 CPU cores. However, DistBelief did not support distributed GPU training at the time of \citet{DistBelief2012}, which could reduce the required resources quite significantly and is in fact used in almost all other implementations mentioned in this section.

\subparagraph{Tensorflow \citep{Tensorflow2015}\citep{Tensorflow2016}}

is the evolution of DistBelief, in the sense that it was developed to replace DistBelief within Google, and borrows both concepts and experience from it. It also applies subsequent optimizations to the parameter server model, like \citet{Chilimbi14} and \citet{Li2014Comms}\citep{Li2014Scaling}. By "both simplifying and generalizing [DistBelief]"\citep{Tensorflow2016}, the Tensorflow developers intend to cater to a wider variety of ideas.

TensorFlow represents both model algorithms and state as a dataflow graph, execution of which can be distributed. This facilitates different parallelization schemes that can take e.g. state locality into account. The level of abstraction of the dataflow graph is that of mathematical operations on tensors (i.e. $n$-dimensional matrices), as opposed to DistBelief, which abstracted at the layer level. Consequently, defining a new type of neural network layer in Tensorflow requires no custom code - it can be represented as a subgraph of a larger model, composed of simpler math operations.

A Tensorflow model is first defined as a symbolic dataflow graph. Once this graph has been constructed, it is optimized and then executed on the available hardware. One advantage of this model is that it allows Tensorflow to tailor its execution strategy towards the types of devices available to it. When working with e.g. GPUs or TPUs (Tensor Processing Units\citep{TPU2017}), Tensorflow can take into account the asynchronicity and intolerance to branching that is inherent to these devices, without requiring any changes to the model itself.

\citet{Shaohuai2017} show Tensorflow achieving sub-40\% efficiency on 4-node, InfiniBand-connected cluster training of ResNet-50\citet{He2015}, and about 75\% efficiency on GoogleNet\citep{Szegedy2014}. 


\subparagraph{MXNet \citep{MXNet2015}}

uses a strategy very similar to that of Tensorflow: models are represented as dataflow graphs, which are executed on hardware that is abstracted away, and coordinated using a parameter server. MXNet, however, also supports imperative definition of dataflow graphs as operations on n-dimensional arrays (which simplifies the implementation of certain kinds of networks).

MXNet's parameter server is used in the form of an abstraction: the KVStore. The KVStore supports pushing key-value pairs from a device to the store, and pulling the current value of a key from the store. There is support for user-defined update logic that is executed when a new value is pushed, and for multiple consistency models enforced by the KVStore (currently limited to sequential and eventually consistent execution). The KVStore is a two-tier system: updates by multiple threads and GPUs are merged on the local machine before they're pushed to the full cluster.

The KVStore abstraction theoretically enables the implementation of (stale-)synchronicity, although only an asynchronous implementation is present at the time of writing. 

On a small cluster (10 machines), MXNet achieves super-linear speedup compared to a single machine when training GoogleNet\citep{Szegedy2014} with more than 10 passes over the data. \citet{Shaohuai2017}, however, show its efficiency to dip to 65\% when trained on more than one machine using GoogleNet\citep{Szegedy2014}, and 50\% using ResNet-50\citet{He2015}. Performance here was evaluated on a single epoch (after warm-up) on nodes linked using Infiniband, but it does suggest that \citet{MXNet2015}'s achieved performance needs some tuning. The paper lacks information on the exact process required.

\subparagraph{CNTK}

again uses a strategy similar to MXNet and Tensorflow. It makes use of dataflow graphs for model definition. Instead of the integrated parameter server approach of the other libraries, however, CNTK uses a separated parameter server called Multiverso, which can also be used by other frameworks.

Performance (both on a single node and on a cluster) seems to be the main factor at which CNTK attempts to differentiate itself. \citet{Shaohuai2017} show CNTK achieving very competitive results of about 75\% efficiency on ResNet-50\citep{He2015} in a 4-node, InfiniBand-connected cluster configuration, outperforming both Tensorflow and MXNet by a significant margin. 

\paragraph{Parallel stale-synchronous SGD}

\subparagraph{Petuum \citep{Xing2013}}

aims to provide a generic platform for any type of machine learning (as long as it is iteratively convergent) on big data (terabytes/petabytes) and big models (hundreds of billions of parameters). It supports data- and model-parallelism. The Petuum approach exploits ML’s error tolerance, dynamic structural dependencies, and non-uniform convergence in order to achieve good scalability on large datasets and models. This is in contrast to e.g. Spark (focusing on fault tolerance \& recovery) and GraphLab (focusing on consistency), which can still be important to ML, but don't inherently have a positive performance impact. The platform uses stale synchronicity to exploit error tolerance (since a minor amount of staleness will have minor effects on convergence), dynamic scheduling policies to exploit dynamic structural dependencies (which helps minimize parallelization error and synchronization cost) and unconverged parameter prioritization to take advantage of non-uniform convergence (by reducing computational cost on parameters that are already near optimal). 

Petuum uses the parameter server model (as proposed by \citet{DistBelief2012}) to keep track of the parameters of the model being trained. The parameter server is also responsible for maintaining the staleness guarantees. In addition, it exposes a scheduler that lets the model developer control the ordering of parallelized model updates.

The programming model that Petuum used is different to that of DistBelief in one core way: data- and model-parallelism operate at the same tier. Whereas in DistBelief model-parallelism is managed inside model shards, and each model replica operates on the parameter server as a single entity, Petuum instead has both model and data slices interact with the parameter server directly, and requires the user to define some central update logic if model-parallelism is involved. This allows slightly more programmer control.

When developing a model using Petuum, developers have to implement at least one method, named push, which is responsible for each of the parallelized model training operations. Its implementation should pull the model state from the parameter server, run a training iteration, and push a gradient to the parameter server. Petuum by default automatically manages the scheduling aspect and the parameter merging logic, so that data-parallel models don’t require any additional operations. If you want model-parallelism, however, you need to also implement the schedule method (which tells each of the parallel workers which parameters to train) and the pull method (which defines the aggregation logic for each of the parallelly generated parameter gradients).

Petuum provides an abstraction layer that also allows it to run on systems using YARN and HDFS, which simplifies compatibility with pre-existing clusters. 


\paragraph{Parallel hybrid-synchronous SGD}

Both synchronous and asynchronous approaches have some significant drawbacks, as is explored by \citet{ChenJianmin2016}. A few frameworks attempt to find a middle ground instead that combines some of the best properties of each model of parallelism, and diminishes some of the drawbacks.

\subparagraph{MXNet-MPI \citep{Mamidala2018}}

takes an approach to distributed ML (using a modified version of MXNet as a proof of concept) that combines some of the best aspects of both asynchronous (parameter server) and synchronous (MPI) implementations. The idea here is to use the same approach as described in the MXNet section, but instead of having single workers communicate with the parameter server, to cluster those workers together into groups that internally apply synchronous SGD over MPI/AllReduce. This has the benefits of easy linear scalability of the sync MPI approach, and fault tolerance of the asynchronous parameter server approach.




\subsubsection{Current challenges}

\paragraph{Performance}

A tradeoff that’s seen frequently is the reduction of wall-clock time at the expense of total processing time (i.e. decreased efficiency). When compute resources are affordable enough, many real-world use cases of machine learning benefit most from being trained rapidly. The fact that this often implies a large multiple in total compute resources (and the associated energy consumption) is not as important, as long as a model saves more money than it costs to run.  A good example of this is found in \citet{DistBelief2012}, where wall clock time speedup factors are achieved by increasing the number of machines quadratically or worse. It still delivered Google competitive advantage for years. Distributed use of GPUs, as in Tensorflow, has better properties, but often still exhibits efficiency below 75\%.
These performance concerns are much less severe in the context of synchronous SGD-based frameworks, which often do achieve linear speedups in benchmarks. However, most of these benchmarks test at most a few hundred machines, whereas the scale at which e.g. DistBelief is demonstrated can sometimes be two orders of magnitude larger. The reason for this is unclear, although it might be related to the fault tolerance concerns mentioned below. Synchronous frameworks also benefit more from high-bandwidth links such as InfiniBand due to the fact that all nodes' communication is synchronized, even though they are expensive (with costs running up to thousands of euros for even a cheap switch) and can't practically be used in large footprint clusters (e.g. in multiple datacenters) due to wiring and latency constraints.

\paragraph{Fault tolerance}

Synchronous AllReduce-based approaches seem to scale significantly better than the parameter server approach (up to a certain cluster size), but suffer from a lack of fault-tolerance: failure of a single machine blocks the entire training process. At smaller scales this might still be a manageable problem, but past a certain number of nodes the probability of any node being unavailable becomes high enough to lead to near-continuous stalling. Common implementations of these HPC-inspired patterns, such as MPI and NCCL, lack fault-tolerance completely, and although there are efforts to counteract some of this, . Some of the described implementations allow for checkpointing to counteract this, but a lot of work is necessary to enable true fault-tolerance, such as described in \citet{Amatya2017}. It is also possible to reduce the probability of failure for each individual node, but this requires very specific hardware that is expensive (e.g. Infiniband) and not generally available on commodity cloud platforms.
Asynchronous implementations do not suffer from this problem quite as much, as they are designed to explicitly tolerate straggling and failing nodes, with only minimal impact on training performance. The question for ML practitioners, then, is whether they prefer performance or fault tolerance, and whether they are constrained by either one. Hybrid approaches even offer a way to customize these characteristics, although they are not frequently found in use yet. It would be interesting to see whether an even better approach exists, or whether there is an efficient way to implement fault-tolerant AllReduce.

\paragraph{Privacy}

There are scenarios in which it is beneficial or even mandatory to isolate different subsets of the training data from each other. The furthest extent of this is when a model needs to be trained on datasets that each live on different machines/clusters, and may under no circumstance be colocated or even moved.
One interesting approach to training a model in a privacy-sensitive context is the use of a distributed ensemble model. This allows perfect separation of the training data subsets, with the drawback that a method needs to be found to properly balance each trained model's output for an unbiased result.
Parameter server-based systems could also be useful in the context of privacy, as the training of a model can be separated from the training's result. However, this assumes that no sensitive properties of the underlying data leak into the model itself, which is not easy to prove.
Finally, it is possible to introduce statistical noise into each subset of the training data, with the intention of rendering its sensitive characteristics unidentifiable to other parties. \citet{Bal12} touches on this subject, but makes it clear that the amount privacy in this scenario is dependent on the amount of statistical queries required to learn the dataset, which puts an upper bound on the usefulness of the model itself.
In conclusion, it is highly unlikely that perfect privacy is possible, but current frameworks do not offer much support for even basic variants. It could be interesting to answer whether there's a generic way to facilitate distributed privacy, which could then be integrated into the frameworks that are being used.

