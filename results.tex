\section{Results}
% Answer research questions

\subsection{The need for Distributed Machine Learning}
\subsubsection{Alternatives to Distributed Machine Learning}
\paragraph{Long-term sustainability}










\subsection{Underlying technology}
To give you an overview of how Distributed Machine Learning works, we'll give you an abstract framework that includes everything a real implementation should include. We do that by exploiting the three unique properties of Distributed Machine Learning, namely error tolerance, dynamic structural dependency and non-uniform convergence.\citet{Xing16}\\
First we'll look at regular Machine Learning algorithms that can be adapted so that they can be used for Distributed Machine Learning.\\
Then we'll look at the algorithms that are used to partition and distribute these algorithms to that they can be used across a wide variety of hardware.
\subsubsection{Machine Learning algorithms}
\paragraph{Overview}
\paragraph{Algorithms to find best parameters}
\subsubsection{Partitioning and distribution algorithms}
\paragraph{Computation time vs communication vs accuracy}
\paragraph{Scheduling and balancing the workloads}
\paragraph{Bridging computation and communication}
\paragraph{How to communicate?}
\paragraph{What to communicate?}










\subsection{Currently used implementations}
\subsubsection{Generic distributed system frameworks}
\subsubsection{Domain-specific implementations}
\subsubsection{Current challenges}










\subsection{Privacy challenges}
