\section{Introduction}
% Convince the reader that Distributed Machine Learning is significant / important / interesting
% Convince the reader that we shouldn't be completely satisfied with the existing literature on this topic and that our research will address some important limitation or deficiency
% Explain our
	% RESEARCH QUESTION
	% SUB-RESEARCH QUESTIONS
	% ALL OF THEM WELL-MOTIVATED
% Give a hypothesis for them

% Clearly formulated andwell-motivated research question. Main research question split up into sub-questions.
% Very well-structured document. General presentation of the content (text and figures) is very effective. 
% Excellent expressed and formulated report. Document has a smooth flow. Spelling and grammatically error free.

The rapid development of new technologies in recent years has lead to an incredible growth of data collection. This creates datasets called Big Data, big enough to require their own set of tools to work with. Machine Learning (ML) algorithms are increasingly being used to analyze datasets. A lot of research is being done on making Machine Learning algorithms work in a distributed environment to deal with Big Data. This paper provides an overview of the current state-of-the-art of Distributed Machine Learning.

Most Machine Learning algorithms assume that all data can be easily accessed. This is not always the case, as some forms of data are inherently distributed or too big to store on single machines, like transaction data for multinationals or astronomical data. Regular Machine Learning algorithms have to be modifier to work with the parallel computation, data distribution and fault detection that is required of distributed algorithms.

% Section [?] describes possible alternatives to Distributed Machine Learning algorithms. Section [?] describes the different types of Machine Learning approaches and their usefulness in a distributed setting. Section [?] describes the currently used implementations and their pros and cons.