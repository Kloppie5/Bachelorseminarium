\section{Introduction}
% Convince the reader that Distributed Machine Learning is significant / important / interesting
% Convince the reader that we shouldn't be completely satisfied with the existing literature on this topic and that our research will address some important limitation or deficiency
% Explain our
	% RESEARCH QUESTION
	% SUB-RESEARCH QUESTIONS
	% ALL OF THEM WELL-MOTIVATED
% Give a hypothesis for them

% Clearly formulated andwell-motivated research question. Main research question split up into sub-questions.
% Very well-structured document. General presentation of the content (text and figures) is very effective. 
% Excellent expressed and formulated report. Document has a smooth flow. Spelling and grammatically error free.

The rapid development of new technologies in recent years has lead to an incredible growth of data collection. Machine Learning (ML) algorithms are increasingly being used to analyze datasets. Most Machine Learning algorithms assume that all data can be easily accessed, but his is not always the case. Some forms of data are inherently distributed or too big to store on single machines, like transaction data for multinationals or astronomical data.

To work with these types of datasets, Machine Learning algorithms have to be modifier to work with the parallel computation, data distribution and fault detection that is required of distributed algorithms. A lot of research is being done on making Machine Learning algorithms work in a distributed environment.

This paper provides an overview of the current state-of-the-art of Distributed Machine Learning. This overview is divided into three parts corresponding to our sub-research questions;
\begin{itemize}
	\item What are the advantages of Distributed Machine Learning over (centralized) Machine Learning?
	\item What technology is used for Distributed Machine Learning?
	\item What are the currently used implementations?
\end{itemize}
Section 4.1 goes more in depth into the need for Distributed Machine Learning; section 4.2 explains the many different types of Machine Learning; and section 4.3 talks about some of the most popular implementations currently in use. Appendix A contains a full overview of the motivation behind each individual subquestion.

% TODO link sections instead of specifically mentioning them