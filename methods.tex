\section{Methods}

\subsection{Study selection}
\subsection{Procedure}
\subsection{Publication bias}
\textbf{These are not my own words!!! Quoted from $ http://journals.plos.org/plosone/article/file?id=10.1371/journal.pone.0096144\&type=printable $}
When performing a meta-analysis, there are concerns that studies with smaller effect sizes are missing (because they are less likely to be reported). Accordingly, an inclusion of unpublished studies in the meta-analysis might have an impact on the average
p-value of all studies combined. A conservative method of addressing this problem is to assume that the effect size of all current or future unpublished studies is equal to zero and to compute the number of such studies it would require to reduce the
overall effect size to a non-significant level (a= .05, two-tailed). To test whether the overall effect that we found was robust, the classic Fail-safe N value as proposed by Rosenthal [54] was calculated.
Rosenthal [54] suggested that findings could be considered as robust if the required number of studies to reduce the overall effect size to a non-significant level exceeded 5 6 number of included studies (effect sizes)+10. Publication bias was also assessed with a
funnel plot visually, and qualified with Egger’s weighted regression method [55]
